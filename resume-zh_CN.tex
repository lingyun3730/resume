% !TEX TS-program = xelatex
% !TEX encoding = UTF-8 Unicode
% !Mode:: "TeX:UTF-8"

\documentclass{resume}
\usepackage{zh_CN-Adobefonts_external} % Simplified Chinese Support using external fonts (./fonts/zh_CN-Adobe/)
%\usepackage{zh_CN-Adobefonts_internal} % Simplified Chinese Support using system fonts
\usepackage{linespacing_fix} % disable extra space before next section
\usepackage{cite}

\begin{document}
\pagenumbering{gobble} % suppress displaying page number

\name{党林涛}

% {E-mail}{mobilephone}{homepage}
% be careful of _ in emaill address
\contactInfo{(+86) 153-1676-8729}{lingyun3730@163.com}{软件工程师}{上海}
% {E-mail}{mobilephone}
% keep the last empty braces!
%\contactInfo{xxx@yuanbin.me}{(+86) 131-221-87xxx}{}

% \section{\faCogs\ IT 技能}
\section{专业技能}
% increase linespacing [parsep=0.5ex]
\begin{itemize}[parsep=0.2ex]
  \item \textbf{支付}: 熟悉支付记账系统架构,以及相关业务包括支付记账,对账,报表生成,以及换汇FX等。
  \item \textbf{数据结构与算法}: 熟练掌握基础数据结构与算法。
  \item \textbf{编程语言}: 掌握Java基础, 微服务框架 Springboot, Java多线程技术等。
  \item \textbf{分布式技术}: 熟悉Raft分布式系统共识协议,日常使用etcd作为分布式系统强一致Key-Value服务注册工具。
  \item \textbf{大数据}: 熟悉Spark编程模型,善于Spark Job性能调优。
  \item \textbf{消息中间件}: 熟悉消息中间件Kafka基本架构和使用。
  \item \textbf{数据库}: 掌握Oracle数据库查询优化,并使用缓存数据库Redis。
  \item \textbf{其他}: 熟悉Kubenetes 基本原理,搭建与使用。
\end{itemize}
% \end{itemize}

\section{工作经历}
\datedsubsection{\textbf{亿贝软件工程(上海)有限公司 | eBay}, Software Engineer - Payments}{2019.04 - 至今}
\begin{itemize}
%   \item 飞猪北京前端团队全面负责各交通线的票务(机票/火车票/汽车票) web 应用与事业群基础架构研发
  \item \textbf{FAS支付记账系统财务报表子系统设计与实现}
  \begin{itemize}
    \item[。] 财务报表子系统是从FAS支付记账系统中查询,分析,聚合,处理用户交易记录信息,并从不同维度实时或者批量生成财务团队需要的报表的平台系统。
    \item[。] 根据财务团队的业务需求,在学习并理解支付记账系统复杂业务逻辑的基础上进行后台开发,后台部分采用支持Fail-Over的Java多线程框架,搭建etcd作为服务注册工具,充分利用集群资源,保证报表生成的效率和准确度。
    \item[。] 负责后台数据与前端对接的Web Service开发,采用基于OIDC的Sail Point身份管理使系统支持基于权限控制的单点登录SSO。
  \end{itemize}
  \item \textbf{支付记账数据 ETL 持久化到Hadoop}
  \begin{itemize}
    \item[。] 对线上数据做抽取,转化和加载,通过Kafka消息中间件,最后将数据以parquet格式持久化到Hadoop HDFS,创建HIVE table schema并做Partition,支持HIVE SQL数据查询。
    \item[。] 创建 batch job 定期对Parquet数据文件合并,并做数据recon。
  \end{itemize}
  \item \textbf{财务报表生成的大数据解决方案}
  \begin{itemize}
    \item[。] 基于HIVE table的数据源,采用基于Scala的Spark 编程来生成支付账户的账龄报告,性能调优显著提升了报告的生成时效。
  \end{itemize}
  \item \textbf{支付系统资金管理平台}
  \begin{itemize}
    \item[。] 根据不同业务需求,支持rule based资金调拨,保证资金调拨的原子性,实时性和幂等性。
  \end{itemize}
  \item \textbf{支付换汇平台搭建}
  \begin{itemize}
    \item[。] 参与搭建FX Center,集成不同Vendor提供的汇率,为公司所有FX换汇业务提供相应的汇率。
    \item[。] 负责FX Core换汇平台搭建,集成公司所有换汇业务的财务相关FX 计算。
  \end{itemize}
  \item \textbf{其他}
  \begin{itemize}
    \item[。] 领导多个内部项目的开发,包括需求讨论,架构设计,开发,自测,Code Review,E2E,并且快速响应功能上线后出现的故障并及时修复,撰写或完善开发文档等。
    \item[。] 与Finance/Accounting同事沟通日常需求,了解他们的痛点后开发了Easy Finance Tool帮助他们更方便准确地在平台上Propose记账需求,支持Approval功能并满足财务Compliance要求。
    \item[。] 2019年8月-2020年3月参与eBay FAS 支付记账系统迁移工作,开发程序对部分账户的余额从老系统到新系统迁移时的对账。
  \end{itemize}
\end{itemize}

\datedsubsection{\textbf{摩根士丹利管理服务(上海)有限公司}, Software Engineer - Summer Intern}{2018.07 - 2018-09}

% \section{\faGraduationCap\ 教育背景}
\section{教育背景}
\datedsubsection{\textbf{上海交通大学},电子与通信工程,\textit{硕士}}{2016.09 - 2019.03}
\ \textbf{GPA: 3.68/4.0}, 获得2018年研究生国家奖学金,2016年全国研究生数学建模竞赛二等奖。
\datedsubsection{\textbf{同济大学},通信工程,\textit{本科}}{2012.09 - 2016.07}
\ \textbf{GPA: 4.8/5.0}, 获得2013,2014,2015年校级学习奖学金,2016年上海市优秀毕业生。

% \begin{onehalfspacing}
% \end{onehalfspacing}

% \datedsubsection{\textbf{DID-ACTE} 荷兰莱顿}{2015年3月 - 2015年6月}
% \role{本科毕业设计}{LIACS 交换生}
% 利用结巴分词对中国古文进行分词与词性标注,用已有领域知识训练形成 classifier 并对结果进行调优
% \begin{onehalfspacing}
% \begin{itemize}
%   \item 利用结巴分词对中国古文进行分词与词性标注
%   \item 利用已有领域知识训练形成 classifier, 并用分词结果进行测试反馈
%   \item 尝试不同规则,对 classifier 进行调优
% \end{itemize}
% \end{onehalfspacing}

%\section{竞赛获奖/项目作品}
%% increase linespacing [parsep=0.5ex]
%\begin{itemize}[parsep=0.2ex]
%%   \item LeetCodeOJ Solutions, \textit{https://github.com/hijiangtao/LeetCodeOJ}
%  \item 第三届中国软件杯大学生软件设计大赛\textbf{全国一等奖}( \textit{http://www.cnsoftbei.com/} ),2014 年8月
%  \item 中国机器人大赛创意设计大赛\textbf{全国特等奖}( \textit{http://www.rcccaa.org/} ),2013年8月
%%   \item 中国机器人大赛暨Robocup公开赛(武术擂台赛)全国一等奖,2013年10月
%  \item 第11届北京理工大学“世纪杯”竞赛学生课外科技作品竞赛\textbf{特等奖},2013年8月
%  \item VIS Components for security system, \textit{https://hijiangtao.github.io/ss-vis-component/}
%  \item 个人博客:\textit{https://hijiangtao.github.io/},更多作品见 \textit{https://github.com/hijiangtao}
%%   \item 电视节目"爸爸去哪儿"可视化分析展示, \textit{https://hijiangtao.github.io/variety-show-hot-spot-vis/}
%\end{itemize}

% \section{\faHeartO\ 项目/作品摘要}
% \section{项目/作品摘要}
% \datedline{\textit{An Integrated Version of Security Monitor Vis System}, https://hijiangtao.github.io/ss-vis-component/ }{}
% \datedline{\textit{Dark-Tech}, https://github.com/hijiangtao/dark-tech/ }{}
% \datedline{\textit{融合社交网络数据挖掘的电视节目可视化分析系统}, https://hijiangtao.github.io/variety-show-hot-spot-vis/}{}
% \datedline{\textit{LeetCodeOJ Solutions}, https://github.com/hijiangtao/LeetCodeOJ}{}
% \datedline{\textit{Info-Vis}, https://github.com/ISCAS-VIS/infovis-ucas}{}

% \section{\faInfo\ 社会实践/其他}
%\section{社区参与/实践其他}
%% increase linespacing [parsep=0.5ex]
%\begin{itemize}[parsep=0.2ex]
%  \item 乐于参与开源社区讨论,\textbf{参与翻译 Vue.js, webpack, WebAssembly, Babel 文档,印记中文成员}
%  \item 中国科学院大学2016秋季学期可视化与可视分析课程助教,\textit{http://vis.ios.ac.cn/infovis-ucas/}
%  \item 未来论坛学生会成员、北理社联新闻信息中心主任、北理工软件学院学生会宣传部副部长(2012-2016)
%  \item 2013-2015 北京市共青团“温暖衣冬”志愿者,第九届园博会志愿者,2014 FLL机器人世锦赛志愿者
%\end{itemize}

%% Reference
%\newpage
%\bibliographystyle{IEEETran}
%\bibliography{mycite}
\end{document}
